\documentclass{beamer}
\newcommand{\myfont}{\rmfamily\normalsize\upshape\mdseries}
\newcommand{\degree}{^\circ}
\newcommand{\R}{\mathbb{R}}
\newcommand{\F}{\mathbb{F}}
\title{\sffamily Review VII(Slides 362 - 443)}
\subtitle{\textbf{Series \& More about Func. Seq}\\ }
\institute[UM-SJTU JI]{University of Michigan-Shanghai Jiao Tong University Joint Institute}
\author{HamHam}
\usepackage{graphicx}
\usepackage{picinpar}
\usepackage{indentfirst}
\usepackage{chemformula}
\usepackage{geometry}
\usepackage{subfigure}
\usepackage{appendix}
\usepackage{amsfonts,amsmath,amssymb}
\usepackage{enumerate}
\usepackage{float}
\usepackage{geometry}
\usepackage{latexsym}
\usepackage{listings}
\usepackage{multicol,multirow,multido}
\usepackage{tabularx}
\usepackage{ulem}
\usepackage{tikz}
\usepackage{xcolor}
\usepackage{cite}
\usepackage{setspace}
\usepackage{textpos}
\usepackage{booktabs}
\usepackage{mathtools, nccmath}
\usepackage{hyperref}
\usetheme[dove]{Boadilla}
\usecolortheme{dolphin}
\useoutertheme{miniframes}

\begin{document}
    \usebackgroundtemplate{\tikz\node[opacity=0.3]{
    \includegraphics[width=\paperwidth,
    height=\paperheight]{hamster.jpg}
    };}
\begin{titlepage}
    \begin{center}
        VV186 - Honors Mathmatics II
    \end{center}
\end{titlepage}
\myfont


\section{Series}


\begin{frame}
    \frametitle{Series}
    \hspace{1em}
    Let $(a_n)$ be a sequence in a normed vector space $(V, ||\cdot||)$. 
    We define $s_n:= \sum_{k=0}^n a_k$  as the \textbf{n-th partial sum} of $(a_n)$. 
    We say that $(a_n)$ is \textbf{summable} with sum $s \in V$ if $\underset{n\to\infty}{\lim}⁡ s_n=s$. 
    We use $\sum_{k=0}^\infty a_k$  , or $∑a_k$ to denote s as well as 
    the "procedure of summing the sequence $(a_n)$", 
    and call this notation \textbf{infinite series}.
    \\
    \vspace{1em}
    Comment. While the definition of series is in general vector space, we will focus on real series and real function series later on. 
\end{frame}

\begin{frame}
    \frametitle{Cauchy Criterion}
    \hspace{1em}
    Generally, a closed form sum of a sequence is hard to find. Instead, we
will mostly focus on whether the series converges.
The starting point will be the \textbf{Cauchy Criterion}(Slides 380):\\
\hspace{1em}Let $\sum a_k$ be a sequence in a complete vector space $(V, ||\cdot||)$ Then  
	\begin{align*}
        \sum a_k \text{ converges}
        &\Leftrightarrow (s_n ) \text{ converges} \\
	    &\Leftrightarrow (s_n) \text{ is Cauchy}\\
	    &\Leftrightarrow \forall \varepsilon>0, \exists N \in \mathbb{N} ~\text{such that}~ \forall m,n>N, ||s_m-s_n ||<\varepsilon\\
	    &\Leftrightarrow \forall \varepsilon>0, \exists N \in \mathbb{N}~ \text{such that}~ \forall m>n>N, ||\sum_{k=n+1}^m a_k ||<\varepsilon 
    \end{align*} 
    \pause
    Two important colloaries are:
        \begin{itemize}
            \item  If $(a_n)$ is summable, $a_n\to 0$ as $n\to \infty$ (and its contraposition)
            \item  If $(a_n)$ is summable, $\sum_{k=n}^\infty a_k\to 0$ as $n \to \infty$
        \end{itemize}
\end{frame}

\begin{frame}
    \frametitle{Tests for Convergence}
\hspace{1em}
A number of tests (for series of positive real numbers) are given
throughout the slides:
\begin{itemize}
    \item 1. The Comparison Test (P393, 3.5.15)
    \item 2. The Root Test (P402, 3.5.22)
    \item 3. The Root Test in Limits Form (P406, 3.5.26)
    \item 4. The Ratio Test (P408, 3.5.28)
    \item 5. The Ratio Test in Limits Form (P411, 3.5.30)
    \item 6. The Ratio Comparison Test (P412, 3.5.31)
    \item 7. Raabe’s Test (P413, 3.5.32)
\end{itemize}



\end{frame}

\begin{frame}
    \frametitle{Procedure of Determining Convergence}
\hspace{1em}We rank the "usefulness" of all these tests as follows:
\begin{center}
\begin{itemize}
    \item[] Cauchy Criteria
    \item[>] Comparison Test
    \item[>] Ratio Test (in Limits)
    \item[>] Root Test (in Limits)
    \item[>] Ratio Comparison Test/Raabe’s Test...  
\end{itemize}
\end{center}
\par
\hspace{1em}
When you are asked to determine whether a series converges, it’s
recommended to use the tests in this order. Thus if you have a hard time
memorizing all the tests, do first memorize the more "important" tests.
    
\end{frame}

\begin{frame}
    \frametitle{Exercises}

    1. Please determine whether the following series converge or not!
    \begin{itemize}
        \item $$\sum_{n=0}^\infty \frac{4n(n+2)!}{(2n)!}$$
        \item $$\sum_{n=1}^\infty \frac{\sin(n\theta)}{n^2} \text{, where } \theta \text{ is fixed}$$%∑2_nϵℕ▒(sin⁡(nθ))/n^2 , where θ is a fixed number
        \item $$\sum_{n=0}^\infty \frac{(2n)!}{4^n (n+1)!n!} \text{ (Hint: This appears in the slides!)}$$%∑_𝑛𝜖ℕ▒((2𝑛)! )/(4^𝑛 (𝑛+1)!𝑛!) 
        \item $$\sum_{n=2}^\infty \frac{1}{(\ln n)^{(\ln n)}}$$%∑2_nϵℕ▒1/(ln⁡n )^ln⁡n   (SJTU Math textbook, P13) 
    \end{itemize}

\end{frame}
\begin{frame}
    \frametitle{Exercises}
    2. Prove the \textbf{limit comparison test}:\\
    \vspace{1em}
    For two positive series $\sum a_n$ and $\sum b_n$, if
    $$0< \underset{n \to \infty}{\lim} \frac{a_n}{b_n}< \infty $$
    then $a_n$ and $b_n$ both converges or diverges.
    \pause \\
    \vspace{0.5em}
    3. Use this result! Prove that if a positive series $a_n$ diverges, then 
    $$\sum \frac{a_n}{1+a_n}$$
    also diverges.
\end{frame}
\begin{frame}
    \frametitle{Absolute and Conditionally Convergence}
    \hspace{1em}
    \begin{itemize}
        \item A series $\sum a_n$ is called \textbf{absolutely convergent} if $\sum ||a_n||$ converges.
        \item If $\sum a_n$ converges while $\sum ||a_n||$ doesn’t, than it’s called \textbf{conditionally convergent}.
        \item In a complete vector space (which is the case in our cases), absolutely
        convergent implies convergent.
    \end{itemize}
To test for conditionally convergence, we have the following theorem:
\hspace{1em}Let $\sum \alpha_k$ be a complex series whose partial sum are bounded but need not converge. 
Let $(a_k)$ be a decreasing convergent sequence with limit zero, then the series $\sum \alpha_k a_k$ converges (Slide 418) 
\\ \vspace{1em}
Comment. With this result, it is easy to see that $\sum_{k=1}^\infty \frac{(-1)^k}{k}$ converges 

\end{frame}
\begin{frame}
    \frametitle{Exercises}
    4. Let $\sum a_k$ be an absolutely convergent real series. Then for any rearrangement 
    $b_j=a_{k_j} , j:\mathbb{N}\to \mathbb{N} $ bijective, $\sum b_j=\sum a_k$. (Slides 417)

    

\end{frame}
\section{Power Series}
\begin{frame}
    \frametitle{Power Series}

   \hspace{1em} Of all function series, one useful kind is the \textbf{power series}, 
    which is the infinite sum of monomials.
    $$\sum_{k=0}^\infty a_k z^k \text{ or simply } \sum a_k z_k$$
    \hspace{1em} We call this \itshape formal \myfont as we are yet to find whether the series converge or
    not for given $z$.\\
    \vspace{1em}
    We can add and multiply two power series:
    \begin{itemize}
        \item $\sum a_n z^n +\sum b_n z^n = \sum (a_n+b_n) z^n$
        \item $\sum a_n z^n \cdot \sum b_n z^n = \sum (a*b)_n z^n$
    \end{itemize}
    \href{https://www.zhihu.com/question/54677157}{\textcolor{red}{\textbf{Why convolution?}}}
\end{frame}
\begin{frame}
    \frametitle{Raduis of Convergence}
    \hspace{1em} Let $\sum a_k z^k$ be a complex power series. 
    Then there exists a unique number $\rho \in (0, +\infty)$ such that 
    \begin{itemize}
        \item[i)] the power series $\sum a_k z^k$ is absolutely convergent at $z_0  \in \mathbb{C}$ if $|z_0 |<\rho$; 
        \item[ii)] the power series diverges at $z_0  \in \mathbb{C}$ if $|z_0 |>\rho$ 
    \end{itemize}
    \vspace{1em}
    Hadamard's formula:
$$\rho=\frac{1}{\underset{k\to \infty}{\varlimsup} \sqrt[k]{|a_k|}}$$
where $\rho$ is called the radius of convergence, if we informally write $1/\infty = 0, 1/0 = \infty$. 

\end{frame}
\begin{frame}
    \frametitle{Remarks}
    \begin{itemize}
        \item For a complex power series, the set of z which the series converge
        will always be a circle. For a real power series, the set will be a line
        segment and the radius of convergence is one half of the length.
        \item We can’t say much if we have $|z| = \rho$. The series may converge or
        diverge or conditionally converge.        
    \end{itemize}
    \vspace{1em}
    Do check for the boundary!
    \begin{figure}
    \centering
    \includegraphics[width=1\textwidth]{boundary.png}
    \end{figure}
\end{frame}
\begin{frame}
    \frametitle{Exercise}
5. Decide for the following real power series, on which interval would it
converge?
\vspace{1em}
\begin{itemize}
    \item $$\sum_{n=1}^\infty \frac{(x-1)^n}{2^n n}$$
    \item $$\sum_{n=1}^\infty \frac{x^{2n}}{n-3^{2n}}$$
\end{itemize}
\end{frame}
\begin{frame}
    \frametitle{Differentiability of Power Series}
    \hspace{1em}
    The power series $\sum a_k z^k$ with radius of convergence $\rho$ 
    defines a differentiable function $f : B_\rho(0) \to \mathbb{C}$. Furthermore,
    $$f~'(z_0)=\sum k a_k z_0^{k-1}$$
    Remarks:
    \begin{itemize}
        \item[1.] This means that we can differentiate a power series ”term by term”
        inside the radius of convergence.
        \item[2.] Recursively apply this theorem to see that any power series is
        infinitely differentiable inside its radius of convergence. In fact, for a
        function to be expressable as a power series (which we call it
        \textbf{analytic}) is stronger than being infinitely differentiable. (You will
        learn more about this in Vv286!)
    \end{itemize}
\end{frame}
\begin{frame}
    \frametitle{Compare and Contrast}
    \centering
    \includegraphics[width=0.6\textwidth]{split3_1.png}
    \includegraphics[width=0.6\textwidth]{split3_3.png}
    \includegraphics[width=0.6\textwidth]{split3_2.png}
\end{frame}
\begin{frame}
    \frametitle{Exercise}
    6. You’ve learnt about Taylor polynomials of a function in your assignments
(see Exercise 7.2). If we let $n \to \infty $ in $ T_n(x; x_0)$, it becomes a \itshape Taylor
Series\myfont. Please use this to find a series representation of
$$f~(x) = \ln (1 + x)$$
around $x = 0$ and determine its radius of convergence.
(You will learn more about Taylor series in the last part of the course)

\end{frame}

\section{More about Func. Seq}

\begin{frame}
    \frametitle{Convergence of Continuous Functions}
    \hspace{1em}    
    Uniform convergence is stronger than pointwise convergence as it
preserves crucial properties of functions such as continuity:
    \begin{figure}[h!]
        \centering
        \includegraphics[width=0.8\textwidth]{cont.png}
    \end{figure}
    Using this, we can prove that $C([a, b])$ is complete:
    \begin{figure}
        \centering
        \includegraphics[width=0.8\textwidth]{complete.png}
    \end{figure}
\end{frame}

\begin{frame}
    \frametitle{Exercise}
    7. Let $f_n(x)$ be an  sequence of functions in $C~^\infty (a,b)$, and 
    $$||f_n~'(x)-f_m~'(x)||\leq A ||f_n(x)-f_m(x)||$$ If $f_n\to f$ as $n\to \infty$
    uniformly, prove that $f_n~' \to f~'$ uniformly. \\(Adapted from vv286 homework)
\end{frame}
\section{Appendix}



\begin{frame}
    \frametitle{Reference}
    \begin{itemize}
        \item Exercises from 2019–Vv186 TA-Zhang Leyang.
        \item Exercises from 2020-Vv186 TA-Xia Yuxuan.
        \item Mathematical Analysis II. \itshape School of Mathematical Sciences, ECNU,\myfont version 5. 
        Beijing: High Education Press, 2019.5 print.
    \end{itemize}
\end{frame}
\begin{frame}
    \frametitle{End}
    \centering
    \LARGE{Thanks!}
    

\end{frame}
\end{document}