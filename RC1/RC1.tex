\documentclass{beamer}
\renewcommand\thesection{\arabic{section}}
\newcommand{\myfont}{\rmfamily\normalsize\upshape\mdseries}
\newcommand{\degree}{^\circ}
\title{\sffamily Review I(Slides 24 - 55)}
\subtitle{\textbf{Logics}\\``Without logic, mathematics will falls apart... "}
\institute[UM-SJTU JI]{University of Michigan-Shanghai Jiao Tong University Joint Institute}
\author{HamHam}
\usepackage{graphicx}
\usepackage{picinpar}
\usepackage{indentfirst}
\usepackage{chemformula}
\usepackage{geometry}
\usepackage{subfigure}
\usepackage{appendix}
\usepackage{amsfonts,amsmath,amssymb}
\usepackage{enumerate}
\usepackage{float}
\usepackage{geometry}
\usepackage{latexsym}
\usepackage{listings}
\usepackage{multicol,multirow,multido}
\usepackage{tabularx}
\usepackage{ulem}
\usepackage{tikz}
\usepackage{xcolor}
\usepackage{cite}
\usepackage{setspace}
\usepackage{hyperref}
\usepackage{textpos}
\usepackage{booktabs}

\usetheme[dove]{Boadilla}
\usecolortheme{dolphin}
\useoutertheme{miniframes}
\begin{document}
    \usebackgroundtemplate{\tikz\node[opacity=0.3]{
    \includegraphics[width=\paperwidth,
    height=\paperheight]{hamster.jpg}
    };}
\begin{titlepage}
    \begin{center}
        VV186 - Honors Mathmatics II
    \end{center}
\end{titlepage}
\myfont
\section{Logics}
\begin{frame}
    \frametitle{Statement}
        First, recall the definition...
        \vspace*{1em}
        \begin{itemize}
            \item statement
            \begin{itemize}
                \item true statement 
                \item false statement
            \end{itemize}
            \item statement structure
            \begin{itemize}
                \item quantifier
                \item statement frame/predicate
                \item specific value
            \end{itemize}
            \item vacuous truth
                \begin{itemize}
                    \item pink elephants could fly!
                    \item more examples?
                \end{itemize}
        \end{itemize}
        \vspace*{1em}
        \myfont
        Easy... but be careful! \\
        \textcolor{red}{Definitions? Examples? Notations?}
\end{frame}

\begin{frame}
    \frametitle{Logical Operation}
    \begin{table}
        \centering
        \resizebox{6cm}{!}{
        \begin{tabular}{c|cc|c}
            \hline \hline
            Type &\multicolumn{2}{c|}{ Logical Operation } & Priority\\
            \hline
            unary& $\neg$&Negation & 1\\ \hline
                 & $\wedge$ &Conjunction & 2 \\ 
                & $\vee$&Disjunction & 3\\
                &  $\Rightarrow$ & Implication & 4\\
                \multirow{-4}{*}{binary}   & $\Leftrightarrow$ & Equivalence & 5\\
            \hline \hline
        \end{tabular}
        }
    \end{table}
    \begin{itemize}
    \item  compound statement
    \begin{itemize}
        \item \itshape tautology
        \item contradiction
        \item contingency
    \end{itemize}
\end{itemize}
\end{frame}
\begin{frame}
    \frametitle{Truth Table}
    \begin{itemize}
        \item Can you draw the truth table of \itshape Implication?
            \begin{itemize}
                \myfont
                \item How to explain F$\Rightarrow$T?
            \end{itemize}
        \myfont
        \item How to use a truth table?
        \begin{itemize}
            \item Understand the problem is about
            \item Cover all the possible situations
            \item e.g. Prove the following using truth table:(see assignments) \\
                \vspace*{1em}
                \* \* \* \* \* \* \* \*
                    $\neg(A\wedge B)\Leftrightarrow(\neg A)\vee(\neg B)$ \* \* \* \* 
                (\textbf{\itshape de Morgen rules})
        \end{itemize}
    \end{itemize}
    \vspace*{5em}
\end{frame}
\begin{frame}
    \frametitle{Relations}
    \begin{itemize}
        \item Proof by Contradiction
        \begin{equation*}
            (A \Rightarrow B) \equiv \neg (A \wedge \neg B)
        \end{equation*}
        \item Proof by Contraposition 
        \begin{equation*}
            (A\Rightarrow B) \equiv (\neg B \Rightarrow \neg A)
        \end{equation*}
    \end{itemize}
    \begin{block}{
    Quick check:}
    \begin{enumerate}
        \item What is \textbf{iff}?
        \item What is \textbf{logically equivalent}? \\
            A: You take the course Vv186. \\
            B: I eat breakfast everyday. \\
            \begin{center}
                A$\Leftrightarrow$B or A$\equiv$B ?        
            \end{center}
    \end{enumerate}
    \end{block}
\end{frame}
\begin{frame}
    \frametitle{Logical Quantifiers}
    \begin{table}
        \centering
        \resizebox{12cm}{!}{%
        \begin{tabular}{ccc}
            \toprule
            \multicolumn{3}{c}{Logical Quantifiers}\\
            \midrule
            Sign & Type & Interpretation\\
            \hline
            $\forall$ & universal & for any; for all\\
            $\exists$ & existential & there exist; there is some\\
            $\forall \dots \forall \dots $ & nesting quantifier& for all \dots for all \dots \\
            $\exists \dots \exists \dots $ & nesting quantifier& there exists \dots (such that) there exist \dots \\
            $\forall \dots \exists \dots $ & nesting quantifier& for any \dots, there exists \dots\\
            $\exists \dots \forall \dots $ & nesting quantifier& there exists \dots (such that) for any \dots\\
            \dots &\dots&\dots\\
            \bottomrule
        \end{tabular}%
        }
    \end{table} 
    \vspace*{1em}
    \begin{itemize}
        \item Try use your own words to interpret them!(see exercises)
        \item Order matters!
    \end{itemize}
\end{frame}
\section{Sets}
\begin{frame}
    \frametitle{Sets}
    \begin{itemize}
        \item What is a set?
        \item Common Type
        \begin{itemize}
            \item Empty set: $\emptyset := \{x:x\neq x\}$
            \item Total set
            \item Subset
            \item Proper subset
            \item Power set(finite)
        \end{itemize}   
        \item Cardinality(finite) 
    \end{itemize}
    \vspace*{1em}
    \begin{block}{
    Quick cheak:}
    \hspace{1em} 
    Let $X=\{x:P(x)\}.$  Is $P(x)$ a statement?
    \end{block}
\end{frame}
\begin{frame}
    \frametitle{Operations on Sets}
    Let
    \begin{equation*}
        A:=\{1,2\} \quad B:=\{2,3\} \quad M:=\{1,2,3,4,5\}
    \end{equation*}
    \begin{table}
        \centering
        \resizebox{6cm}{!}{%
        \begin{tabular}{ccc}
            \toprule
            \multicolumn{3}{c}{Set Operations}\\
            \midrule
            $A\cup B$ & Union & $\{1,2,3\}$\\
            $A\cap B$ & Intersection & $\{2\}$\\
            $A\setminus B$ & Difference & $\{1\}$\\
            $A^c $ & Complement & $\{3,4,5\}$\\
            \bottomrule
        \end{tabular}%
        }
    \end{table}  
    \begin{itemize}
        \item The notation $A - B $ is also used for $A\setminus B$ and $\bar{A}$ for $A^c$
    \end{itemize}
\end{frame}
\begin{frame}
    \frametitle{Ordered Pairs}
\begin{itemize}
    \item What is an ordered pair?
    \item What is the difference between ordered pair and set?
    \item Concept of \itshape{Cartesian product}\myfont.
\end{itemize}
\vspace{1em}
$$A \times B := \{(a,b):a \in A, b \in B\}$$
$$\mathbb{R}^2=\mathbb{R} \times \mathbb{R}$$
\vspace*{7em}
\end{frame}
%\begin{frame}
%    \frametitle{Conceptual diagram}
%    \usetikzlibrary{mindmap}
%    \begin{tikzpicture}[mindmap, concept color=blue!60]
%        \node [extra concept] {Statement}
%            child[concept color=red!50,grow=30] {node[concept] (c1) {Statement structure}}
%            child[concept color=orange!50,grow=0] {node[concept] {vacuous truth}};
%        \node [extra concept] at (3,-2) {Sets}
%            child[concept color=green!50, grow=0] {node[concept] {operations}};
%        \node [extra concept] (c2) at (8,1) {Logical operators};
%        \node [extra concept] (c3) at (9,3) {Truth Table};
%        \begin{pgfonlayer}{background}
%            \draw [extra concept connection] 
%                                           
%                                        (c2) edge (c3);
%            \end{pgfonlayer}
%    \end{tikzpicture}
%\end{frame}
\section{Exercises}
\begin{frame}
    \frametitle{Exercises}
    1. Let $A, B, C$ be three statements. Use truth table to prove that 
    \begin{equation*}
        (A\vee B)\wedge C \equiv (A\wedge C)\vee (B\wedge C)
    \end{equation*}
    2. Let $A, B, C$ be three sets. Prove that
    \begin{equation*}
        (A\cup B)\cap C =(A\cap C)\cup (B\cap C)
    \end{equation*}
    \begin{itemize}
        \item From Exercise 1 \& 2, we can see that sets and statements are similar.
        \item Venn diagram?
    \end{itemize}
\end{frame}
\begin{frame}
\frametitle{Exercises}
3. Check whether the following sentences are true statement, false statement, or not a statement.
\begin{itemize}
    \item $\forall x, y\in \mathbb{R}, x^2+y^3\geq 0$
    \item Let  $f(a)=a^4$, then $f(0)>0$
    \item For any $a\in \mathbb{R}, a^4>0$
    \item An African Elephant is very big.
    \item Let $A, B$ be two statements, then $(A\vee B)\Leftrightarrow\neg(\neg A\wedge\neg B)$
\end{itemize}
\end{frame}
\begin{frame}
    \frametitle{Exercises}
4. Use quantifiers to rewrite the following definition of covergence:
\par \vspace{2em} \hspace{1em} Let $(a_n)_{n\in \mathbb{N}}$ be a real sequence. If for some fixed $a \in \mathbb{R}$, 
for any $\varepsilon>0$, there is an $N\in \mathbb{N}$, such that for all $n>N$, $|a_n-a|<\varepsilon$,\\ then we
say $(a_n)$ converges to $a$.\\
\vspace{2em}
What's the negation of this statement?
\end{frame}
\begin{frame}
    \frametitle{Exercises}
    5. Interpret the following statement with your own words:\\
    \begin{equation*}
        \forall x(H(x)\vee \exists y(H(y)\wedge F(x,y)))
    \end{equation*}    
    Here $H(x)$ means "$x$ takes Vv186", $F(x,y)$ means "$x$ and $y$ are friends", 
    both $x$ and $y$ represent a freshman student in JI.
\end{frame}
\begin{frame}
    \frametitle{Exercises}
    6$^*$. Sheffer stroke \& Peirce arrow (also see in assignments)\\[1em]
    p NAND q: false iff p\&q are both true, marked as $p$ | $q$ \\
    p NOR q : \* true iff p\& are both false, marked as $p\downarrow q$\\[1em]
    Do the following: \\
    \begin{itemize}
        \item Draw the truth table of NAND and NOR
        \item Prove that $p\downarrow q$ is logically equivalent to $\neg(p\vee q)$
        \item Prove taht $p$ | $q$ is equivalent to $q$ | $p$
        \item Use the operator $\downarrow$ only to construct the statement $p\Rightarrow q$
    \end{itemize}
\end{frame}
\begin{frame}
    \frametitle{Reference}
    \begin{itemize}
        \item Exercises from 2019--Vv186 TA-Zhang Leyang.
        \item Exercises from 2020--Vv186 TA-Xia Yuxuan.
        \item Exercises from 2020--Vv186 TA-Hu Pingbang.
        \item Kenneth, H.Rosen. Translated by Xu Liutong etc. \itshape Discrete Mathematics amd Its Applications\myfont, 
                 Eightth Edition, Chinese Abridgement. China Machine Press, 2019 print.
    \end{itemize}
\end{frame}
\end{document}