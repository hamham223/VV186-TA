\documentclass{beamer}
\renewcommand\thesection{\arabic{section}}
\newcommand{\myfont}{\rmfamily\normalsize\upshape\mdseries}
\newcommand{\degree}{^\circ}
\title{\sffamily Review VIII Practical Integral(Slides 555-567)}
\subtitle{\textbf{Integrate!Integrate!Integrate!}\\}
\institute[UM-SJTU JI]{University of Michigan-Shanghai Jiao Tong University Joint Institute}
\author{HamHam}
\usepackage{graphicx}
\usepackage{picinpar}
\usepackage{indentfirst}
\usepackage{chemformula}
\usepackage{geometry}
\usepackage{subfigure}
\usepackage{appendix}
\usepackage{amsfonts,amsmath,amssymb}
\usepackage{enumerate}
\usepackage{float}
\usepackage{geometry}
\usepackage{latexsym}
\usepackage{listings}
\usepackage{multicol,multirow,multido}
\usepackage{tabularx}
\usepackage{ulem}
\usepackage{tikz}
\usepackage{xcolor}
\usepackage{cite}
\usepackage{setspace}
\usepackage{hyperref}
\usepackage{textpos}
\usepackage{booktabs}

\usetheme[dove]{Boadilla}
\usecolortheme{dolphin}
\useoutertheme{miniframes}
\begin{document}
    \usebackgroundtemplate{\tikz\node[opacity=0.3]{
    \includegraphics[width=\paperwidth,
    height=\paperheight]{hamster.jpg}
    };}
\begin{titlepage}
    \begin{center}
        VV186 - Honors Mathmatics II
    \end{center}
\end{titlepage}
\myfont

\begin{frame}
    \frametitle{Integration}
    \centering
    \includegraphics[height=0.85\textheight]{integral.jpg}
\end{frame}
\begin{frame}
    \frametitle{Can you find the original formula?}
    \centering
    \includegraphics[height=0.8\textheight]{fourier.jpg}
\end{frame}
\begin{frame}
    \frametitle{Integration Method}
    Method0: Symmetry\\
    \vspace{1em}\hspace{1em}
    Suppose $f~(x)$ is an odd integrable function, then $$\int_{-a}^a f~(x)dx=0$$
    \\
    \vspace{1em}
    \hspace{1em} Exercises: For $a>0$, calculte 
    $$\int_{-a}^a \frac{\cos(x)}{1+e(x)^{o(x)}} dx$$
    where $e(x)$ is a continuous strictly positive even function, and $o(x)$ is an odd function
\end{frame}
\begin{frame}
    \frametitle{Integration Method}
    Method1: Recite!\\
    \vspace{1em}
    Common indefinite integrals include:
    \begin{itemize}
        \item $\int x^\alpha dx=\frac{x^{\alpha+1}}{\alpha+1} +C$
        \item $\int e^x dx=e^x +C $
        \item $\int \frac{1}{x}dx=\ln(|x|)+C$
        \item $\int \cos x~ dx = \sin x +C$
        \item $\int \sin x~ dx = -\cos x +C$
        \item $\int \ln(x)= ?$
    \end{itemize}
    For more complex integrals, we need other theorems to help us evaluate
them.
\end{frame}
\begin{frame}
    \frametitle{Exercise}
    Calculte the following integrals:
    \begin{itemize}
        \item $$\int \frac{2}{\sqrt{x}^3} dx$$
        \item $$\int_3^7 \frac{1}{x^2+6x+5} dx$$
    \end{itemize}
    \vspace{1em}
    \hspace{1em}
    \itshape
    Comment. Partial fraction is sometimes powerful!\myfont
\end{frame}
\begin{frame}
    \frametitle{Integration Method}
    Method2: Substitution!\\
    \vspace{1em}
    \begin{itemize}
        \item Let $u=g(x)$.
        \item Compute $du=g'(x)$.
        \item Substitute $g(x)=u$ and $g'(x)=du$. \textbf{At this moment, only $u$, no $x$ !!!}
        \item Calculte the above integral of $u$, it should be easier.
        \item Replace $u$ by $g(x)$ to get the result with $x$.
        \item For definite integral, pay attention to the range.
    \end{itemize}
    \begin{block}{Demo}
        $$1. \int sin(x) cos(x) dx ~~~~~~~~~~~~~~~~~2.\int \frac{1}{1+x^2} dx$$
    \end{block}
\end{frame}
\begin{frame}
    \frametitle{Exercise}
    Calculte the following integrals:
    \begin{itemize}
        \item $$\int_{-1}^3 \sqrt{9-x^2} dx$$
        \item $$\int \tan(x) dx$$
        \item $$\int \frac{x}{3x^2+6x+10} dx$$
        \item $$\int \frac{e^{4x}}{1+e^{2x}} dx$$
    \end{itemize}
\end{frame}
\begin{frame}
    \frametitle{Integral Method}
    Method3: Integration by part!\\
    \vspace{1em}

    For definite integral:
    $$\int_a^b f~'(x)g(x) dx=f~(x)g(x) \Big|_a^b-\int_a^b f~(x)g'(x) dx$$
    For indefinite integral:
    $$\int f~'(x)g(x) dx=f~(x)g(x) -\int f~(x)g'(x) dx$$
    Demo:
    $$\int x \sin (x) dx$$
\end{frame}
\begin{frame}
    \frametitle{Exercise}

    \begin{itemize}
        \item $$\int x^2 e^{-x} dx$$
        \item $$\int (\ln x)^2 dx$$
        \item $$\int_0^{\frac{\pi}{2}} \sin^n x ~dx$$
    \end{itemize}

\end{frame}
\begin{frame}
    \frametitle{Methodology}
    \textbf{\textcolor{red}{1. Substitute what? When to subsitute?}}
    \begin{block}{Depend on your luck!}
        \begin{itemize}
            \item $\sqrt{a- x^2} \mapsto \text{ use } x=\sqrt{a}\sin(u)$
            \item $x^2+a^2 \mapsto \text{ use } x=a\tan (u)$
            \item $x^2-a^2 \mapsto \text{ use } x=a \csc(u)$
            \item Similar terms? Complex terms?
            \item Other method? Partial fraction?
        \end{itemize}
    \end{block}
    \textbf{\textcolor{red}{2. Integrate by part: which to integrate, which to differentiate?}}
    \begin{block}{Practice makes perfect!}
        \begin{itemize}
            \item Easy and simple terms --> integrate.
            \item DI table.
        \end{itemize}
    \end{block}
\end{frame}
\begin{frame}
    \frametitle{Integral Method}
    Method4: DI Table! (Enhanced version of Method3)
    \\
    \vspace{3em}\hspace{1em}
    \href{https://www.youtube.com/watch?v=2I-_SV8cwsw}{\large{\textcolor{blue}{DI Table Method!}}}
    (Youtube link, VPN required) \\
    \hspace{1em} Link: \url{https://www.youtube.com/watch?v=2I-_SV8cwsw}
    \vspace{1em}
    \\\hspace{1em}
This is an explanation video on DI method, you can also find other
interesting and useful math videos on this channel. After watch the video, calculte
$$\mathcal{L}\sin(x)=\int_0^{\infty} e^{-px} \sin(bx) dx$$
This is called the \itshape Laplace Transform.\myfont
\end{frame}
\begin{frame}
    \frametitle{Some challenge!}
    More Integration Method? Take VV286!
    \vspace{2em}
    $$\int \frac{1}{x^4} dx \longmapsto \int \frac{1}{x^4+4}dx \longmapsto \int_{-\infty}^{\infty} \frac{1}{x^4+1} dx$$
\end{frame}

\begin{frame}
    \frametitle{Farewell}
    \begin{itemize}
        \item Congratulation! You are almost done! Thanks for your hard-working!
        \item I'm really glad to be your TA, thanks for your support! 
        \item Perhaps I can't be your Vv285 TA... but    wish you all the best in your future life 
        and find your own way in JI!
        \pause
    \vspace{1em}\\
        After VV186: (there are more math classes waiting you)
    \end{itemize}
        \begin{itemize}
        \item Honors Mathematics Sequence: \textcolor{gray}{VV186}-VV285-VV286
        \item \textcolor{gray}{Linear Algebra: VV214 / VV417}
        \item Discrete Mathematics: VE203
        \item \textcolor{gray}{Probabilistic Methods in Engineering: VE401}
        \item \textcolor{gray}{Partial Differential Equations: VV557}
    \end{itemize}
    \hspace{1em}
    \tiny{another choice is taking vv255!}
\end{frame}
\begin{frame}
    \frametitle{A simple question¿}
    \LARGE
    Every now and then, ask yourself: \\
    \vspace{1em}
    \hspace{1em}
    what is “math” based on all the math you have learned in your life?
    \\\hspace{4em}--question from my TA's TA


\end{frame}
\begin{frame}
    \frametitle{Reference}
    \begin{itemize}
        \item Exercises from 2020--Vv186 TA-Xia Yuxuan.
        \item Exercises from Vv286 Assignment and Slides.
        \item Picture on the door of Dr. Horst Hohberger's Office.
        \item Mathematical Analysis I. \itshape School of Mathematical Sciences, ECNU,\myfont version 5. 
        Beijing: High Education Press, 2019.5 print.
        \item Exercise from JI first integration bee.
    \end{itemize}
\end{frame}
\end{document}